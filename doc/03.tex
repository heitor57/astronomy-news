\section{Modelo Conceitual Banco De Dados}

Para modelar conceitualmente o Banco de Dados foi utilizado o Modelo Entidade-Relacionamento Estendido (Modelo EER). A princípio foram definidos alguns elementos do tipo entidade e do tipo relacionamento que seriam essenciais para a estrutura do Banco de Dados, como por exemplo:

\begin{itemize}
				\item Tipo Entidade: Portal, Publicação, Pessoa, Autor, Comentário; 

				\item  Tipo Relacionamento: Publicação ESTA\_EM Portal, Publicação ESCRITA\_POR Autor, Pessoa ESCREVE Comentário;
\end{itemize}

Através da análise detalhada do minimundo, e refinamentos no modelo pré-estabelecido, obteve-se o modelo mostrado na Figura 1.


\subsection{Tipos Entidades}

Os Tipos Entidades presentes no modelo da Figura 1 estão descritos a seguir, junto aos seus atributos:

\begin{itemize}
				\item PORTAL: Tipo Entidade que representa um portal de notícias;

\begin{itemize}
\item URL: atributo chave que representa o endereço de rede do portal;
\item Logo: atributo simples que representa a imagem do logotipo do portal;
\item Nome: atributo simples que representa o nome do portal.
\end{itemize}
\end{itemize}
\begin{itemize}
\item PUBLICACAO: Tipo Entidade que representa uma publicação/notícia.
\begin{itemize}
\item Palavras\_chave: atributo simples multivalorado que representa as palavras relacionadas a publicação;
\item Título: atributo simples que representa o título da publicação;
\item Subtítulo: atributo simples que representa a manchete/descrição rápida da publicação;
\item Tempo\_leitura: atributo simples que representa o tempo de leitura estimado da publicação;
\item Imagem\_principal: atributo simples que representa a imagem de chamada da publicação;
\item Texto: atributo simples que representa o conteúdo textual da publicação;
\item Ultima\_modificacao: atributo simples que representa a última modificação realizada na publicação;
\item URL: atributo chave que representa a URL da publicação;
\item Timestamp\_obtencao: atributo que representa o momento em que a publicação foi indexada pelo crawler.

\end{itemize}
\end{itemize}
\begin{itemize}

\item PESSOA: Tipo Entidade que representa uma pessoa no minimundo;
\begin{itemize}
\item URL: atributo chave que representa uma URL relacionada à pessoa;
\item Nome: atributo simples que representa o nome da pessoa;
\item Imagem\_perfil: atributo simples que representa a foto de perfil da pessoa.
\end{itemize}
\end{itemize}
\begin{itemize}
\item AUTOR: Tipo Entidade que representa o escritor da publicação. É uma especialização de pessoa;
				\begin{itemize}
\item Palavras\_chave: atributo simples multivalorado que representa as palavras chaves das publicações que o autor escreve;
\item Descricao: atributo simples que representa uma breve descrição sobre o autor.
				\end{itemize}
				\end{itemize}
				\begin{itemize}
\item COMENTARIO: Tipo Entidade que representa um comentário na publicação.
				\begin{itemize}
\item ID: atributo chave que representa um identificador para o comentário;
\item Texto: atributo simples que representa o texto do comentário por uma pessoa;
\item Timestamp\_escrita: atributo simples que representa o momento que o comentário foi feito.
				\end{itemize}
				\end{itemize}
\subsection{Tipos Relacionamentos}

Os tipos relacionamentos presentes no modelo da Figura 1 podem ser descritos em função dos Tipo Entidades envolvidos no relacionamento, da razão de cardinalidade e da restrição de parcialidade. A seguir será apresentado as informações de cada Tipo Relacionamento exposto no modelo EER.

\begin{itemize}
				
\item ESTA\_EM

Descreve a relação entre publicações e um portal, no qual uma publicação deve pertencer a somente um portal, mas um portal pode ter várias publicações.

Tem-se, então, as seguintes propriedades:

\begin{itemize}

\item Entidades envolvidas: PUBLICACAO, PORTAL
\item Razão de cardinalidade: 1 do lado de PUBLICACAO; N do lado de PORTAL
\item Restrição de Parcialidade: Total do lado de PUBLICACAO                        Parcial do lado de PORTAL
\item Como se lê: Publicação está em Portal; Portal possui Publicação
\end{itemize}
\end{itemize}

\begin{itemize}

				\item ESCRITA\_POR

Descreve a relação entre autores/escritores e publicações, no qual um autor pode escrever nenhuma ou várias publicações, e uma publicação pode ter nenhum, um ou mais de um autor (existem casos nos domínios usados onde não é indicado o autor da publicação).

Tem-se, então, as seguintes propriedades:

\begin{itemize}

				\item Entidades envolvidas: PUBLICACAO, AUTOR
				\item Razão de cardinalidade:     N do lado de PUBLICACAO;  N do lado de AUTOR
				\item Restrição de Parcialidade: Parcial do lado de PUBLICACAO                        Parcial do lado de AUTOR
				\item Como se lê: Publicação escrita por Autor; Autor escreve Publicação
\end{itemize}
\end{itemize}

\begin{itemize}
				\item POSSUI

Descreve a relação entre uma publicação e comentários feitos na publicação, no qual uma publicação pode conter nenhum ou vários comentários, enquanto que um comentário necessariamente deve estar contido em uma publicação.

Tem-se, então, as seguintes propriedades:
\begin{itemize}
				\item Entidades envolvidas: PUBLICACAO, COMENTARIO
				\item Razão de cardinalidade:     N do lado de PUBLICACAO
                1 do lado de COMENTARIO
				\item Restrição de Parcialidade: Parcial do lado de PUBLICACAO                        Total do lado de COMENTARIO
				\item Como se lê: Publicação possui Comentário; Comentário está em Publicação
\end{itemize}
\end{itemize}

\begin{itemize}
				\item ESCREVE

Descreve a relação entre uma pessoa e os comentários feitos por esta pessoa, no qual uma pessoa pode fazer de nenhum à vários comentários, e por outro lado, um comentário é feito por uma única pessoa.

Tem-se, então, as seguintes propriedades:
\begin{itemize}

\item Entidades envolvidas: PESSOA, COMENTARIO
\item Razão de cardinalidade: N do lado de PESSOA; 1 do lado de COMENTARIO
\item Restrição de Parcialidade: Parcial do lado de PESSOA                        Total do lado de COMENTARIO
\item Como se lê: Pessoa escreve Comentario; Comentario é escrito por Pessoa
\end{itemize}
\end{itemize}

\begin{itemize}
				\item RESPONDE

Descreve a relação entre comentários, no qual um comentário pode ser respondido por nenhum, um ou vários outros comentários, e um comentário deve necessariamente ser escrito em resposta a nenhum (no caso do início de uma thread) ou a exclusivamente um comentário (no caso de continuação de uma thread).

Tem-se, então, as seguintes propriedades:
\begin{itemize}

\item Entidades envolvidas: COMENTARIO, COMENTARIO (self-loop)
\item Razão de cardinalidade:     N do lado de COMENTARIO                        ; 1 do lado de COMENTARIO
\item Restrição de Parcialidade: Parcial do lado de COMENTARIO                        parcial do lado de COMENTARIO
\item Como se lê: Comentário responde comentário; Comentário é respondido por Comentário
\end{itemize}
\end{itemize}

\subsection{Especializações}

Durante a elaboração do modelo EER foi proposto um relacionamento superclasse/subclasse para os tipos entidades.

Este relacionamento superclasse/subclasse acontece entre o tipo entidade PESSOA (superclasse) e o tipo entidade AUTOR (subclasse). Este relacionamento veio da necessidade de especializar o tipo entidade PESSOA na subclasse AUTOR, dado que certos atributos e relacionamentos podem se aplicar a algumas, mas não a todas as entidades da superclasse PESSOA.

 As propriedades gerais deste relacionamento estão descritas a seguir:
 \begin{itemize}
				 \item Superclasse: PESSOA
				 \item Subclasse: AUTOR
				 \item Processo adotado: Especialização
				 \item Restrição de completude: Parcial
 \end{itemize}
