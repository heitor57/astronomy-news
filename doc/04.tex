
\section{Conclusão}

Diante da descrição do coletor de dados e do modelo EER do banco de dados discutido neste documento, nota-se que há campo para avançar para a próxima etapa deste projeto, que é o desenvolvimento de uma aplicação web que reúna os dados coletados permitindo diversas consultas a estes dados de forma estruturada.

Os dados obtidos até então por meio do crawler foram 111 publicações e 151 pessoas, mostrando que o nicho de astronomia possue um certo nível de interações de usuários, que pode nos favorecer para fazer análises de descobrir pontos interessantes nesse domínio de notícias. Tópicos em alta recentemente como por exemplo terra plana, ida do homem a lua, robos em solo marciano podem ter grande impacto nesse número de pessoas por número de publicações. Porém essa ainda é so uma amostragem do que iremos obter com mais tempo de desenvolvimento de projeto.

As principais dificuldades encontradas dizem respeito ao contraste de informações presentes em cada domínio em particular, o que resultou numa coleta não uniforme de dados, visto que algumas informações podem estar ausentes em determinadas publicações e presentes em outras. Outra dificuldade ainda relacionado a coleta de dados, foi a conversão do conteúdo principal da publicação, o conteúdo textual, pois pode haver diversos tipos de estruturas entre os parágrafos da publicação, como subtítulos, players de áudio, vídeo, carrossel de imagens, postagens do twitter entre outros, levando ao contratempo de converter, estruturar e representar todos estes componentes.


Nos próximos passos, pretende-se aplicar técnicas de mineração de dados, para encontrar padrões consistentes nos dados coletados permitindo, como por exemplo, a classificação da publicação em subcategorias. Estuda-se também, manusear algoritmos de aprendizado de máquina para inferir novas informações, como a predição do tempo de leitura da publicação. Além da aplicação de métodos de análise sentimental nos comentários dos internautas através do processamento de linguagem natural, entre outros procedimentos que possibilitem a obtenção das mais variadas informações acerca dos dados coletados.

Por fim com uma interface gráfica, que pretendemos construir, será possível apresentar todo esse conteúdo incluindo as análises de maneira elegante.
