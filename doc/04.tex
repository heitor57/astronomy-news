
\section{Conclusão}
Diante do promissor desenvolvimento do coletor de dados e do modelo EER do banco de dados discutido neste documento, nota-se que há campo para avançar para a próxima etapa deste projeto, que é o desenvolvimento de uma aplicação web que reúna os dados coletados permitindo diversas consultas a estes dados de forma estruturada.

Pretende-se aplicar técnicas de mineração de dados, permitindo a inferência de novas informações, como por exemplo a análise de sentimentos dos comentários dos internautas, a predição do tempo de leitura da publicação, a classificação da publicação em sub-temas, processamento de linguagem natural, entre muitos outros.  

As principais dificuldades encontradas dizem respeito aos diferentes componentes presentes em cada domínio em particular, o que resultou numa coleta não uniforme de dados, visto que algumas informações podem estar ausentes em determinadas publicações e presentes em outras. A coleta do conteúdo principal da publicação merece destaque neste quesito, visto que existe a possibilidade de conter players de áudio, vídeo, carrossel de imagens, postagens do twitter entre outros.
