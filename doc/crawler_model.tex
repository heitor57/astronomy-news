
\begin{tikzpicture}[every node/.style={text width=2cm,align=center},comp/.style={draw,circle},node distance=5.5cm]
				\node[draw, rectangle] (1) {Usuário interno};
				\node[comp, below of= 1, yshift=2.5cm] (2) {Carrega a página da URL};
				\node[comp, below left of= 2] (3) {Extrai Links};
				\node[comp, below right of= 2] (4) {HTML Parser};
				\node[comp, below of= 2] (5) {URL List};
				\draw [vecArrow] (1) -- node[xshift=0.9cm] {URL Seed} (2);
				\draw [vecArrow] (2) -|node[xshift=0.5cm,yshift=0.7cm,] {Página de navegação baixada} (3);
				\draw [vecArrow] (3) --node[xshift=0.4cm,yshift=-0.4cm,] {Links} (2);
				\draw [vecArrow] (2) -| node[xshift=-0.5cm,yshift=0.7cm] {Página de publicação baixada} (4);
				\draw [vecArrow] (4) -- node[xshift=-0.4cm,yshift=-0.4cm,] {Dados estruturados} (2);
				\draw [vecArrow] (2) --node[xshift=0.6cm,yshift=-0.8cm] {URL} (5);
\end{tikzpicture}

%\tikz[line width=1.1pt,font={\fontsize{9pt}{12}\selectfont},every node/.style={draw,circle,text width=2cm,align=center},every edge/.style={latex-}]
%\graph [random seed=10,layered layout,sibling distance=2.5cm,grow'=down,]
%{
				%"Usuário interno" ->[edge label=Seed URL] {"Carrega a página da URL" -> {"Extrai Links", "Lista de URLs", "HTML Parser"} },
				%{"Extrai Links", "Lista de URLs", "HTML Parser"} -> "Carrega a página da URL"
%};
