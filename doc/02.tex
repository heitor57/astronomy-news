\section{Modelo Conceitual Banco De Dados}

Para modelar conceitualmente o Banco de Dados foi utilizado o Modelo Entidade-Relacionamento Estendido (Modelo EER). A princípio foram definidos alguns elementos do tipo entidade e do tipo relacionamento que seriam essenciais para a estrutura do Banco de Dados, como por exemplo:

\begin{itemize}
				\item Tipo entidade: Portal, Publicação, Pessoa, Autor, Comentário; 

				\item  Tipo Relacionamento: Publicação ESTA\_EM Portal, Publicação ESCRITA\_POR Autor, Pessoa ESCREVE Comentário;
\end{itemize}

Através da análise detalhada do minimundo, e refinamentos no modelo pré-estabelecido, obteve-se o modelo mostrado na Figura 1.

\begin{figure}[H]
				\centering
\documentclass{standalone}
\usepackage{tikz}
\usetikzlibrary {arrows.meta,graphs,graphdrawing}
\usegdlibrary {layered}
\usetikzlibrary {graphs,graphdrawing,quotes} \usegdlibrary {force}
\usetikzlibrary {graphs,graphdrawing} \usegdlibrary {trees}
\usetikzlibrary{arrows}
\usetikzlibrary{shapes}
\usetikzlibrary{er}
\usetikzlibrary{quotes}
\usepackage[T1]{fontenc}
\catcode`\_=12 % change catcode of "_" to "other" (no. 12)
\begin{document}
\newbox\underlinebox


%\tikzset{entity/.style={draw,rectangle}}
\tikzset{every relationship/.style={text width=2cm,align=center}}

\tikzset{total/.style={double distance=1.5pt}}
\tikzset{derived attribute/.style={attribute,dashed}}
%\tikzset{attribute/.style={draw,ellipse}}
\tikzset{multi attribute/.style={attribute,double distance=1.5pt}}
\tikzset{
				key attribute/.append style={
								font=\rmfamily,
								execute at begin node={%
												\setbox\underlinebox=\hbox\bgroup
								},
								execute at end node={%
												\egroup\underline{\box\underlinebox}%
								}
				}
}
\tikzset{pics/supsetarc/.style={code={
				\draw (-0.1,0.15) to[out=0,in=0,looseness=3]    (-0.1,-0.15);}},
				pics/subsetarc/.style={code={
				\draw (0.1,0.15) to[out=180,in=180,looseness=3]    (0.1,-0.15);}},
				supset/.style={to path={(\tikztostart) -- 
				pic[pos=0.5,sloped,allow upside down] {supsetarc} (\tikztotarget)}},
				subset/.style={to path={(\tikztostart) -- 
pic[pos=0.5,sloped,allow upside down] {subsetarc} (\tikztotarget)}}}


\tikzset{
				%every attribute/.append style={fill=black!10,draw=black},
				every relationship/.append style={fill=black!5,draw=black},
				every entity/.append style={fill=black!5,draw=black},
%every entity/.append style={fill=blue!20,draw=blue,thick},
%every relationship/.append style={fill=orange!20,draw=orange,thick,aspect=1.5},
}

\tikz[line width=1.1pt,font={\fontsize{9pt}{12}\selectfont}]
\graph [random seed=10,layered layout,node distance=8.5cm]
{
				PUBLICACAO[entity] -- {
								Titulo[attribute], 
								Tempo_leitura[attribute],
								Ultima_modificacao[attribute], 
								Palavras_chave[multi attribute], 
								Comentarios[multi attribute], 
								Imagem_principal[attribute], 
								Subtitulo[attribute],
				},
				AUTOR[entity] -- {
				NomePessoa[as=Nome,attribute], AUTOR_URL[as=URL,key attribute], Descrição[attribute], pPalavras_chave[as=Palavras_chave,attribute], Imagem_perfil[attribute]},
				PAGINA[entity] -- {
								PAGINAURL[as=URL,key attribute], Timestamp_obtencao[attribute], Timestamp_processamento[attribute], Seed[attribute]
				},
				PORTAL[entity] -- {
								Logo[attribute], NomePortal[as=Nome,attribute], URL[key attribute],
				},
				PAGINA --[total,edge label=1,pos=0.95] "ESTA_EM"[relationship] --[edge label=N,pos=0.05] PORTAL,
				PUBLICACAO --[edge label=N,pos=0.95] "ESCRITA POR"[relationship] --[edge label=N,pos=0.05] AUTOR,
				PAGINA --[supset] PUBLICACAO,
};

\end{document}

\caption{Modelo EER.}
\end{figure}
\subsection{Tipos Entidades}

Os tipos entidades presentes no modelo da Figura 1 estão descritos a seguir, junto aos seus atributos:

\begin{itemize}

				\item PORTAL: Tipo entidade que representa um portal de notícias;

								O tipo entidade Portal é importante para identificação da origem da extração dos dados, a URL é utilizada como identificador assim como em outros tipo entidade devido as caracteristícas desse minimundo. Logo e nome vão ser atributos importantes na fase da implementação da apresentação dos dados obtidos na coleta de dados.
\begin{itemize}
\item URL: atributo chave que representa o endereço de rede do portal;
\item Logo: atributo simples que representa a imagem do logotipo do portal;
\item Nome: atributo simples que representa o nome do portal.
\end{itemize}
\end{itemize}
\begin{itemize}
\item PUBLICACAO: Tipo entidade que representa uma publicação/notícia.

				O tipo entidade PUBLICACAO possue grande variedade de atributos, alguns deles são para apresentação da notícia como por exemplo Título, Subtítulo, Imagem\_principal. Alguns atributos são metadados referentes a uma publicação (PUBLICACAO), como por exemplo tempo de leitura (Tempo\_leitura), ultima modificação (Ultima\_modificacao). Por fim utilizamos um atributo como metadado referente a coleta dos dados que é o Timestamp\_obtencao, esse atributo foi criado, pois é importante o registro do tempo de obtenção para decisões futuras sobre atualização da base entre outras razões.
\begin{itemize}
\item Palavras\_chave: atributo simples multivalorado que representa as palavras relacionadas a publicação;
\item Título: atributo simples que representa o título da publicação;
\item Subtítulo: atributo simples que representa a manchete/descrição rápida da publicação;
\item Tempo\_leitura: atributo simples que representa o tempo de leitura estimado da publicação;
\item Imagem\_principal: atributo simples que representa a imagem de chamada da publicação;
\item Conteudo: atributo simples que representa o conteúdo principal da publicação;
\item Ultima\_modificacao: atributo simples que representa a última modificação realizada na publicação;
\item URL: atributo chave que representa a URL da publicação;
\item Timestamp\_obtencao: atributo que representa o momento em que a publicação foi indexada pelo crawler.

\end{itemize}
\end{itemize}
\begin{itemize}

\item PESSOA: Tipo entidade que representa uma pessoa no minimundo;

				Esse é um tipo entidade importante para definição, pois modela pessoas diversas que interagem nesse minimundo por comentários (poderia ser outros meios) e não possuem atributos e relações que um autor de publicações possui.
\begin{itemize}
\item URL: atributo chave que representa uma URL relacionada à pessoa;
\item Nome: atributo simples que representa o nome da pessoa;
\item Imagem\_perfil: atributo simples que representa a foto de perfil da pessoa.
\end{itemize}
\end{itemize}
\begin{itemize}
\item AUTOR: Tipo entidade que representa o escritor da publicação. É uma especialização de pessoa;

				AUTOR é uma especialização de PESSOA como dito anteriormente, possui um relacionamento com PUBLICACAO, e além disso possui metadados para descrição do autor de diferentes formas por meio dos atributos Palavras\_chave, Descricao.
				\begin{itemize}
\item Palavras\_chave: atributo simples multivalorado que representa as palavras chaves das publicações que o autor escreve;
\item Descricao: atributo simples que representa uma breve descrição sobre o autor.
				\end{itemize}
				\end{itemize}
				\begin{itemize}
\item COMENTARIO: Tipo entidade que representa um comentário na publicação.

				Esse tipo entidade representa um comentário que uma pessoa faz. Esse comentário é destinado a uma publicação, e em certos casos é uma resposta a outros comentários.
				\begin{itemize}
\item ID: atributo chave que representa um identificador para o comentário;
\item Texto: atributo simples que representa o texto do comentário por uma pessoa;
\item Timestamp\_escrita: atributo simples que representa o momento que o comentário foi feito.
				\end{itemize}
				\end{itemize}
\subsection{Tipos Relacionamentos}

Os tipos relacionamentos presentes no modelo da Figura 1 podem ser descritos em função dos tipo entidades envolvidos no relacionamento, da razão de cardinalidade e da restrição de parcialidade. A seguir será apresentado as informações de cada tipo relacionamento exposto no modelo EER.

\begin{itemize}
				
\item ESTA\_EM

Descreve a relação entre publicações e um portal, no qual uma publicação deve pertencer a somente um portal, mas um portal pode ter várias publicações.

Tem-se, então, as seguintes propriedades:

\begin{itemize}

\item Entidades envolvidas: PUBLICACAO, PORTAL;
\item Razão de cardinalidade: PUBLICACAO:PORTAL tem razão de cardinalidade N:1;
\item Restrição de Parcialidade: Total do lado de PUBLICACAO; Parcial do lado de PORTAL;
\item Como se lê: Publicação está em Portal; Portal possui Publicação;
\end{itemize}
\end{itemize}

\begin{itemize}

				\item ESCRITA\_POR

Descreve a relação entre autores/escritores e publicações, no qual um autor pode escrever nenhuma ou várias publicações, e uma publicação pode ter nenhum, um ou mais de um autor (existem casos nos domínios usados onde não é indicado o autor da publicação).

Tem-se, então, as seguintes propriedades:

\begin{itemize}

				\item Entidades envolvidas: PUBLICACAO, AUTOR;
				\item Razão de cardinalidade: PUBLICACAO:AUTOR tem razão de cardinalidade N:N;
				\item Restrição de Parcialidade: Parcial do lado de PUBLICACAO; Parcial do lado de AUTOR;
				\item Como se lê: Publicação escrita por Autor; Autor escreve Publicação.
\end{itemize}
\end{itemize}

\begin{itemize}
				\item POSSUI

Descreve a relação entre uma publicação e comentários feitos na publicação, no qual uma publicação pode conter nenhum ou vários comentários, enquanto que um comentário necessariamente deve estar contido em uma publicação.

Tem-se, então, as seguintes propriedades:
\begin{itemize}
				\item Entidades envolvidas: PUBLICACAO, COMENTARIO;
				\item Razão de cardinalidade: PUBLICACAO:COMENTARIO tem razão de cardinalidade 1:N;
				\item Restrição de Parcialidade: Parcial do lado de PUBLICACAO; Total do lado de COMENTARIO;
				\item Como se lê: Publicação possui Comentário; Comentário está em Publicação
\end{itemize}
\end{itemize}

\begin{itemize}
				\item ESCREVE

Descreve a relação entre uma pessoa e os comentários feitos por esta pessoa, no qual uma pessoa pode fazer de nenhum à vários comentários, e por outro lado, um comentário é feito por uma única pessoa.

Tem-se, então, as seguintes propriedades:
\begin{itemize}

\item Entidades envolvidas: PESSOA, COMENTARIO;
\item Razão de cardinalidade: PESSOA:COMENTARIO possui razão de cardinalidade 1:N;
\item Restrição de Parcialidade: Parcial do lado de PESSOA; Total do lado de COMENTARIO;
\item Como se lê: Pessoa escreve Comentario; Comentario é escrito por Pessoa.
\end{itemize}
\end{itemize}

\begin{itemize}
				\item RESPONDE

Descreve a relação entre comentários, no qual um comentário pode ser respondido por nenhum, um ou vários outros comentários, e um comentário deve necessariamente ser escrito em resposta a nenhum (no caso do início de uma thread) ou a exclusivamente um comentário (no caso de continuação de uma thread).

Tem-se, então, as seguintes propriedades:
\begin{itemize}

\item Entidades envolvidas: COMENTARIO, COMENTARIO;
\item Razão de cardinalidade: COMENTARIO:COMENTARIO 1:N;
\item Restrição de Parcialidade: Parcial de ambos lados;
\item Como se lê: Comentário responde comentário; Comentário é respondido por Comentário
\end{itemize}
\end{itemize}

\subsection{Especializações}

Durante a elaboração do modelo EER foi proposto um relacionamento superclasse/subclasse para os tipos entidades.

Este relacionamento superclasse/subclasse acontece entre o tipo entidade PESSOA (superclasse) e o tipo entidade AUTOR (subclasse). Este relacionamento veio da necessidade de especializar o tipo entidade PESSOA na subclasse AUTOR, dado que certos atributos e relacionamentos podem se aplicar a algumas, mas não a todas as entidades da superclasse PESSOA.

 As propriedades gerais deste relacionamento estão descritas a seguir:
 \begin{itemize}
				 \item Superclasse: PESSOA
				 \item Subclasse: AUTOR
				 \item Processo adotado: Especialização
				 \item Restrição de completude: Parcial
 \end{itemize}
